\section{Diskussion}
\label{sec:Diskussion}

Es werden mit den drei Methoden verschiedene Werte für die Zeitkonstante errechnet.
Mit dem Aufladeprozess wird 

\begin{center}
    $RC = 1.26 \symup{s} \pm 0.8 \%$
\end{center}

bestimmt. Mit der Amplitude in Abhängigkeit von der Frequenz wird 

\begin{center}
    $RC = 1.36 \symup{s} \pm 1.5 \%$
\end{center}

und mittels der Phasenverschiebung in Abhängigkeit der Frequenz wird

\begin{center}
    $RC = 1.33 \symup{s} \pm 3.8\%$
\end{center}

bestimmt. Da diese Werte alle nah beieinander liegen kann von einer im Allgemeinen guten Messung ausgegangen werden.
Die Messwerte liegen auch, bis auf zwei bei der Phasenverschiebungsmessreihe, sehr nahe der Theoriekurven.
In besagter Messreihe der Phasenverschiebung werden die Messwerte bei geringer Frequenz ungenau, was auch gut in dem Polarplot in \autoref{fig:polar} zu sehen ist.
Zunächst sind die Werte nahezu exakt auf der Theoriekurve während sie gegen Ende immer mehr Abweichen. Die nicht betrachteten Messwerte geben sogar eine Phasenverschiebung an, die nicht zwischen $0$ und $\frac{\pi}{2}$ liegt.

Aufgrund des Innenwiderstandes des Funktionsgenerators existiert ein gewisser systematischer Fehler, der in der Auswertung allerdings nicht beachet wird.

Die Integrationsfunktion des RC-Kreises ist gegeben, da die entstehenden Funktionen sehr genau den erwarteten entsprechen.