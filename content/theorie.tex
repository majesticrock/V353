\section{Zielsetzung}
Ziel des Versuches ist es, einen RC-Schwingkreis, also einen Stromkreis, welcher im wesentlichen aus einem Kondensator und einem Widerstand 
besteht, auf sein Relaxationsverhalten hin zu untersuchen. Die Untersuchung beschäftigt sich mit der Bestimmung der Zeitkonstante
und der Frequenzabhängigkeit (einer angelegten Wechselspannung) der Amplitude beziehungsweise Phasendifferenz. Anschließend soll der
RC-Kreis auf seine Eigenschaft als Integrator hin untersucht werden.

\section{Theorie}
\label{sec:Theorie}

Bei der nicht-oszillatorischen Rückkehr eines vom 
Ruhezustand ausglenekten Systems in den Selbigen
kann es zu Relaxationserscheinungen kommen. Die Zeit,
welche das System hierzu benötigt, wird als 
Relaxationszeit bezeichnet. Im Allgemeinen kann 
für eine physikalische Größe $A$ die Änderungsgeschwindigkeit
im Zeitpunkt $t$ als 
\begin{equation}
\label{eqn:aenderungsgeschwindigkeit}
\frac{\symup{d} \, A} {\symup{d} \, t} = c (A(t) - A(\infty))
\end{equation}
angegeben werden, wobei $A(\infty)$ den Endzustand bezeichnet. 
Daraus folgt mittels Integration vom Zeitpunkt $t = 0$ bis $t$
\begin{equation}
\label{eqn:A}
A(t) = A(\infty) + [ A(0) - A(\infty)] \cdot e^{ct},
\end{equation}
wobei $c < 0$ gelten muss, damit $A(t)$ beschränkt ist.
Das Relaxationsverhalten wird im Versuch durch den 
Entlade - beziehungsweise Ladevorgang eines mit einem 
Widerstand verbundenen Kondensator verwirklicht.

\subsection{Der Aufladevorgang}
    Die Spannung eines Kondensators lässt sich mit Hilfe der gespeicherten Ladung $Q$ und der Kapazität $C$ des Kondensator als 
    \begin{equation}
        \label{eqn:UC}
        U_C = \frac{Q}{C}
    \end{equation}
    darstellen. Nach dem Ohm'schen Gesetz folgt, wenn der Kondensator über einen Widerstand verbunden ist, dass
    \begin{equation}
        \label{eqn:strom}
        I = \frac{U_C}{R},
    \end{equation}
    wobei $I$ die Stromstärke ist. Hieraus folgt für die Änderung der Ladung pro Zeiteinheit:
    \begin{equation}
        \label{eqn:ladung}
        \frac{\symup{d} \, Q}{\symup{d} \, t} = - \frac{1}{RC} Q(t).
    \end{equation}
    Da hier ein Aufladevorgang betrachtet wird, ist der Kondensator am Anfang ungeladen und am Ende ungeladen, also gelten
    \\ \\
    \centerline{$Q(0) = 0 \text{ und } Q(\infty)  = C U_0$}
    \\ \\
    Mittels Gleichung \eqref{eqn:A} folgt daraus die Gleichung für den Aufladevorgang zu
    \begin{equation}
        \label{eqn:aufladung}
        Q(t) = C U_0 (1 - e^{-\frac{t}{RC}} )
    \end{equation}
    und damit
    \begin{equation}
        \label{eqn:wunschformel}
        U(t) = U_0 (1 - e^{-\frac{t}{RC}} ),
    \end{equation}    
    wobei $RC$ die Zeitkonstante des Relaxationsvorganges ist.    
 
\subsection{Relaxationsverhalten bei angelegten Wechselspannungen}
    Durch das Anlegen einer Wechselspannung 
    \begin{equation}
        \label{eqn:Wechselspannung}
        U(t) = U_0 \cdot \cos(\omega t)
    \end{equation}
    bildet sich eine Phasenverschiebung zwischen der eingehenden Wechselspannung und der vom Kondensator 
    ausgehenden Wechselspannung, welche sich mit zunehmender Frequenz verzögert. Mit dem Ansatz 
    \begin{equation}
        \label{eqn:Kondensatorspannung}
        U_C (t) = A(\omega) \cdot \cos(\omega t + \phi(\omega))
    \end{equation}
    wird die Kondensatorspannung $U_C$ beschrieben, wobei $\phi$ die Phasenverschiebung ist.
    Mit dem Zusammenhang
    \begin{equation}
        \label{eqn:stromstaerke}
        I(t) = \frac{\symup{d} \, Q} {\symup{d} \, t} = C \frac{\symup{d} \, U_C} {\symup{d} \, t},
    \end{equation}
    sowie dem zweiten Kirchhoff'schen Gesetz ergibt sich die Phasendifferenz $\phi$ zu 
    \begin{equation}
        \label{eqn:phasendifferenz}
        \phi(\omega) = \arctan (-\omega R C).
    \end{equation}
    Als Funktion der Periodendauer $b$ und dem Abstand der Nulldurchläufe zweier phasenverschobener Spannungen $a$, ergibt sich die Phasendifferenz $\phi$
    zudem zu
    \begin{equation}
        \label{eqn:phasenmessung}
        \phi = \frac {a}{b} \cdot 2 \pi.
    \end{equation}
    Hieraus lässt sich die Frequenzabhängigkeit der Phasendifferenz feststellen, sowie das Verschwinden der Phasendifferenz für niedrige 
    Frequenzen beziehungsweise die Konvergenz der Phasendifferenz für hohe Frequenzen gegen $\frac{\pi}{2}$.
    Mit dem ebenfalls resultierenden Zusammenhang von Amplitude und Phasendifferenz
    \begin{equation}
        \label{eqn:zusammenhang}
        A(\omega) = - \frac {\sin(\phi)}{\omega R C} U_0,
    \end{equation}
    ergibt sich die Amplitude $A$ zu
    \begin{equation}
    \label{eqn:amplitude}
    A(\omega) = \frac{U_0}{\sqrt{1 + \omega^2 R^2 C^2}}.
    \end{equation}

\subsection{Der RC-Schwingkreis als Integrator}
    Der RC-Kreis hat die Eigenschaft unter gewissen Vorraussetzungen eine zeitlich veränderliche Spannung $U(t)$ zu intergrieren. 
    Aus dem zweiten Kirchhoff'schen Gesetz und Gleichung \eqref{eqn:stromstaerke} folgt
    \begin{equation}
        \label{eqn:spannungint}
        U(t) = R C \frac{\symup{d} \, U_C} {\symup{d} \, t} + U_C(t).
    \end{equation}
    Unter den Vorraussetzung $\omega >> \frac{1}{R C}$, ist $\lvert U_C \rvert << \lvert U_R \rvert $ und $\lvert U_C \rvert << \lvert U \rvert $, 
    sodass sich näherungsweise sagen lässt:
    \begin{equation}
        \label{eqn:int1}
        U(t) = R C \frac{\symup{d} \, U_C} {\symup{d} \, t}
    \end{equation}
    beziehungsweise
    \begin{equation}
        \label{eqn:int2}
        U_C(t) = \frac {1} {R C} \int_0^{t} U(t') \symup{d} \, t'.    
    \end{equation}    